% !TeX root = RJwrapper.tex
\newcommand{\factorMergerTitle}{\href{https://github.com/geneticsMiNIng/factorMerger}{factorMerger}}
\newcommand{\factorMerger}{\href{https://github.com/geneticsMiNIng/factorMerger}{factorMerger }}
\definecolor{orange}{rgb}{1,0.5,0}
\newcommand{\todo}{\textcolor{red}}
\newcommand{\M}{\mathcal{M}}

\title{\factorMergerTitle: A Set of Tools to Support Results From Post
Hoc Testing}
\author{by Agnieszka Sitko and Przemysław Biecek}

\maketitle

\abstract{
\emph{ANOVA}-like statistical tests for differences among groups are available for almost a hundred years. However, for large number of groups the results from commonly used \emph{post-hoc} tests are often hard to interpret. To deal with this problem, the \factorMerger  package constructs and plots the hierarchical relation for the considered factor. Such a~hierarchical structure is derived based on the~\emph{Likelihood~Ratio~Test} and is presented with the~\emph{Factor~Merger~Tree} created with the~\textbf{ggplot2} package~\citep{ggplot2}. The~current implementation handles one-dimensional and multi-dimensional Gaussian models as well as binomial and survival models. In this article the~methodological background is outlined and the~main functionalities of the packege are
illustrated using real-data examples.
}

\section{Introduction}

In this article we present the \factorMerger package that enriches results from \emph{ANOVA} tests.  The~\emph{ANOVA} method verifies the null hypothesis that a variable of interest $y$ has the same distribution in all groups that are being compared.
If this null hypothesis is rejected a more detailed analysis of differences among categorical variable levels might be needed. The traditional approach is to perform \emph{pairwise post hoc tests} in order to verify which groups differ significantly. 

One may find implementations of the traditional \emph{post hoc tests} in many \emph{R} packages. Package \textbf{agricolae} \citep{Agric} offers a wide range of them. It gives one of the most popular \emph{post hoc test}, Tukey HSD test (\code{HSD.test}), its less conservative version --- Student-Newman-Keuls test (\code{SNK.test}) or Scheffe test (\code{scheffe.test}) which is robust to factor imbalance. These parametric tests are based on Student's t-distribution, thus, are reduced to Gaussian models only. In contrasts, \textbf{multcomp} package \citep{Multcomp} can be used with generalized linear models (function \code{glht}) as it uses general linear hypothesis. Similarly to the \textbf{multcomp}, some implementations that accept \code{glm} objects are also given in \textbf{car} (\code{linearHypothesis}, \citealp{car}) and \textbf{lsmeans} \citep{lsmeans}.

However, an undeniable disadvantage of single-step \emph{post hoc tests} is the inconsistency of their results. For a fixed significance level, it is possible that mean in group A does not differ significantly from the one in group B, similarly with groups B and C. At the same time the difference between group A and C is detected. Then data partition is unequivocal and, as a consequence, impossible to put through. 

The problem of clustering categorical variable into non-overlapping groups has already been present in the literature. First, J. Tukey proposed an iterative procedure of merging factor levels based on the studentized range distribution \citep{Tukey}. However, statistical test used in this approach made it limited to Gaussian models. \emph{Collapse And Shrinkage in ANOVA} (\emph{CAS-ANOVA}, \citealp{Casanova}) is an algorithm that extends categorical variable partitioning for generalized linear models in testing. It is based on the Tibshirani's \emph{Fused LASSO} \citep{Tib} with the constraint taken on the pairwise differences within a factor, which yields to their smoothing.

\emph{Delete or Merge Regressors} algorithm (\citealp{Proch}, p. 37) is also adjusted to generalized linear models. It directly uses the hierarchical clustering to gain hierarchical structure of a factor. At~the~beginning, \emph{DMR4glm} calculates the likelihood ratio test statistics for models arising from pairwise merging of factor levels or deleting factor levels against the full model (the one with all groups included). Then it performs agglomerative clustering taking LRT statistic as a distance --- each step of clustering is associated with a model with different factor structure. Experimental studies (\citealp{Proch}, p. 44--91) showed that the \emph{Delete or Merge Regressors}'s performance is better than \emph{CAS-ANOVA}'s when it comes to the model accuracy. \emph{Delete or Merge Regressors}'s implementation may be found in the \textbf{DMR} package \citep{DMR}.

In this article we present a more direct approach to the problem of merging groups that are being compared. The \factorMerger package offers an algorithm of hierarchical clustering of factors based on a backward iterative procedure. In each step it chooses a model with the highest \emph{Likelihood~Ratio~Test} test p-value or, in other words, the highest likelihood. While this algorithm is more complex than \emph{DMR4glm}, it~maximizes the likelihood on the merging path\footnote{Although it may be shown that the \emph{DMR} algorithm is a consistent
model selection method, its performance on smaller datasets is undefined. \todo{TODO....}}. What is more, it is easily expandable for non-parametric models (using permutation tests instead of \emph{LRT}s). 

In addition to the comprehensive algorithm which tries to merge all possible pairs of levels in a step, also a \emph{successive version} is provided. In the \emph{successive version} only levels which are relatively close can be merged (levels distance is dependent on the model chosen). While the basic approach (all vs. all comparisons) may result in a slightly better partition from the statistical point of view, proposed extension (all vs. subsequent comparisons) seems to be more graceful when it comes to the interpretation. Moreover, the former algorithm is more computationally expensive.

Furthermore, the \factorMerger package gives an approximate implementation of \emph{DMR4glm} (skipping the deleting procedure) with its \emph{successive version}.

More detailed description of all algorithms implemented in \factorMerger is given in the section \nameref{algs}.

\section{Methodology}\label{algs}

Merging procedures implemented in the \factorMerger package begin with the full model --- with all levels of a given factor included --- and iteratively merge one pair of levels until the factor is constant. Uniting two groups reduces by one the number of subsets, so, as initially we have finite number of levels, the procedure will eventually obtain one-level-factor and terminate. In a single iteration pairs \emph{worth uniting} are considered and the one which optimizes some objective function is joined. Objective functions use likelihood-based statistics. We will specify them later on. 

The \factorMerger package gives the ability to perform analysis for the wide family of models and choose from the broad spectrum of merging approaches. Depending on the problem statement, some parts of the merging procedure may differ. The general sketch of~the~algorithm is described below.

\begin{algorithm}[H]
\caption{The outline of the merging procedure}
\label{outline}
\begin{algorithmic}[2]

\Function{MergeFactors}{$response, factor, family,
successive, method$}
\State{$pairsSet := generatePairs(response, factor, successive$)}
\State{$\mathcal{M}:= createModel(response, factor, family)$}
\While{$|levels(factor)| > 1$}
    \State{$toBeMerged := \mathrm{argmax_{pair \in pairsSet}}objectiveFunction(pair, response, factor)$}    
    \State{$\mathcal{M} := updateModel(\mathcal{M},
   \; toBeMerged)$}
    \State{$factor := mergeLevels(factor, pair)$}
    \State{$pairsSet := joinPair(pairsSet, pair) $}
\EndWhile
    \EndFunction
\end{algorithmic}
\end{algorithm}

In the article we will denote:
\begin{itemize}

\item the sample size as $n$,
\item the initial number of factor levels as $k$,
\item the binary matrix representing group membership as $X = \{x_ij\}_{i,j=1}^{n,k}$. 

$$
x_{ij} = \left\{
                \begin{array}{ll}
                  1\;\; \mathrm{if} \; i \mathrm{-th \;
                  observation \; belongs \; to 
                  \; group\;} j, \\
                  0 \;\; \mathrm{otherwise.} 
                \end{array}
       \right.$$
We assume that groups do not overlap.
\item the response vector as $y = (y_1, \ldots, y_n)$ or the response matrix as $Y = \{y_{ij}\}_{i,j=1}^{n,m}$,
\item the effects --- vector of a length $k$  or $k \times m$ matrix --- as $\beta$.

\end{itemize}

\break

\subsection{Model family}

In the current version the package supports parametric models: 

\begin{itemize}
\item single dimensional Gaussian (with the argument \code{family = "gaussian"}),
\item multi dimensional Gaussian --- Gaussian model with multiple $y$ outputs (with the argument \code{family = "gaussian"})\footnote{Both single dimensional and multi dimensional Gaussian models use \code{family = "gaussian"}. However, multi~dimensional model uses different functions for likelihood estimation and may require additional preprocessing, thus, it is considered as a~separate category.},
\item binomial (with the argument \code{family = "binomial"}),
\item survival (with the argument \code{family = "survival"}).
\end{itemize}

Each case has its own method of estimating model parameters and a specific likelihood formula.

\paragraph{Single dimensional Gaussian model}

A convenient and commonly made assumption in the analysis of variance is the normality of the errors. Here we will consider a linear model in which beta coefficients represent group means. The model may be written in vector form as

$$y = X \beta + \epsilon, \;\; \epsilon \sim \mathcal{N}\left(0, \sigma^2\right).$$


Under the above assumptions we may formulate the likelihood of the Gaussian linear model (\citealp{friedman2001elements}, p.31)

$$L\left(\beta, \sigma | y\right) = \left(2\pi \sigma^2\right)^{-\frac{n}{2}} 
\exp{\left(-\frac{1}{2}\left(y - X\beta\right)^T\left(y - X\beta\right)/ \sigma^2\right)}$$

and the corresponding logarithm of the likelihood

$$l\left(\beta, \sigma | y\right) = 
-\frac{n}{2} \log{\left(2\pi\right)} -\frac{n}{2} \log{\left(\sigma^2\right)} -\frac{1}{2}\left(y - X\beta\right)^T\left(y - X\beta\right)/ \sigma^2.$$


To calculate the loglikelihood in the package we use \code{logLik.lm\{stats\}}.


\paragraph{Multi dimensional Gaussian model} In the multi dimensional generalization of the Gaussian model we will observe multiple outputs $Y$. It takes the following form

$$Y = X \beta + E, \;\; E \sim \mathcal{N}(0, \Sigma),$$

where $\beta = \{\beta\}_{i,j=1}^{k,m}$ is an $k \times m$ effects matrix and $E = \left(\epsilon_1, \epsilon_2, ..., \epsilon_m\right)$ is an $m$-dimensional error.

Now, we may write the likelihood 
$$L\left(\beta, \Sigma | Y\right) = \left(|2\pi \Sigma|\right)^{-\frac{1}{2}} 
\exp{\left(-\frac{1}{2}\left(Y - X\beta\right)^T\Sigma^{-1}\left(Y - X\beta\right)\right)}$$

and its logarithm

$$l\left(\beta, \sigma | Y\right) = 
-\frac{n}{2} \log{\left(2\pi\right)} -\frac{n}{2} \log{\left(|\Sigma|\right)} -\frac{1}{2}\left(Y - X\beta\right)^T\Sigma^{-1}\left(Y - X\beta\right).$$

Unfortunately, \textbf{stats} or any commonly used \emph{R} package do not support multiple responses in the loglikelihood calculation for linear Gaussian models. In the package we use \code{logLik.mlm} implementation introduced in the \textbf{Atools} package \citep{atools} and the \code{dmvnorm\{mvtnorm\}} \citep{mvtnorm} implementation for multivariate normal density estimation.

\break

\paragraph{Binomial model}

In the binomial case the observed response $y$ will take one of two possible disjoint outcomes (success or failure). We will be interested whether probabilities of success in given subpopulations are statistically the same. Let us introduce the assumption
$$y \sim \mathcal{B}\left(p, n\right)$$

where $\mathcal{B}\left(p,n\right)$ is the binomial distribution with the probability of success $p$ and the number~of~trials~$n$. We consider the logit model 

$$\ln\left(\frac{p}{1 - p}\right) = X \beta.$$

Let $z = \sum_{i = 1}^n y_i$. Now the likelihood may be written as follows  \citep{binom}

$$L\left(\beta | y \right) = 
\frac{n!}{z!\left(n - z\right)!}p ^z \left(1 - p\right)^{n - z}.$$

Thus, the logarithm of the likelihood is expressed as 

$$l\left(\beta|y\right) = zX\beta - n \log{\left(1 + \exp^{X\beta}\right)}.$$

\todo{TODO: Policzyć loglik, czy na pewno dobrze.}

To calculate the loglikelihood for the binomial model we use \code{logLik.glm\{stats\}}.

\paragraph{Survival model}

Survival analysis is a branch of statistics for analyzing the time until a~specified event takes to happen (such as a patient death or a machine failure). Usually we assume that survival time of an individual depends on two factors: the underlying baseline hazard function and a set of explanatory variables. In the package for survival analysis applications the \emph{Cox proportional hazard regression} introduced by D. R. Cox \citep{coxph} is used. In this model we assume that the baseline hazard function is the same for the whole population and the effects of predictors are multiplicatively related to the individual hazard. 

Let $\lambda(t)$ be the hazard function and let $\lambda_0(t)$ be the baseline hazard function, where $t$ denotes time. Then the Cox model has the following form

$$\lambda(t) = 
\lambda_0(t)\cdot \exp{(X\beta)}.$$

Here $y = (y_1, \ldots, y_n)$ is a vector of observed times in the sample. For $i$-th observation $y_i$ can be either event time or censoring time. We say that an observation is censored if we do not have complete information about its status in the context of our interest (for example, the patient died  of causes other than the disease of our consideration or was lost to follow-up). Let $C = (C_1, \ldots, C_n)$ be a vector of indicators that the time corresponds to the event. To be more presice,

$$C_i = \left\{
                \begin{array}{ll}
                  1 \;\; \mathrm{if \; for} \; i 
                  \mathrm{-th \; 
                  observation \; the \; event \; occured,} \\
                  0 \;\; \mathrm{if} \; i \mathrm{-th \;
                  observation \; was \; censored.}
                \end{array}
              \right.$$

Then we may construct the partial likelihood (as a function of $\beta$ only) in the following way

$$L(\beta|y) = 
	\prod_{i: C_i = 1}
		\left[
			\frac{\exp{\left(X_i\beta\right)}}
					{\sum_{j: y_j \geq y_i} 
					\exp{\left(X_i\beta\right)}}
			\right]$$
			
and the corresponding loglikelihood
$$
l\left(\beta\right|y) = 
	\sum_{i:C_i = 1} 
		\left(
			X_i\beta -
			\log{
				\left(
				\sum_{j:y_j\geq y_i} 
				\exp{\left(X_i\beta\right)}		
				\right)
			}
		\right).
$$

\todo{TODO: Dlaczego możemy korzystać z częściowej wiarygodności?}

The Cox regression is implemented in the \textbf{survival} package (\code{coxph\{survival\}}, \citealp{survival-book})
and the loglikelihood of the model one may find by accessing the field \code{loglik} of a \code{coxph.object}.

\break

\paragraph{The Likelihood Ratio Test statistics}

The substantial part of \factorMergerTitle 's algorithms is calculating the \emph{Likelihood Ratio Test statistics}. In this paragraph we will formulate its definition.

Let us assume that the initial factor variable divides population into $l$ subgroups and denote~the effect of~a~group $i$ on~the~response as $\beta_i$. We define $h_{ij}$, a constraint on~groups $i$~and~$j$~claiming that their group effects are statistically the same, as

\begin{equation} \label{constraint}
h_{ij}: \beta_i = \beta_j, \;\; i \neq j \;\; i,j \in \{1, 2, ..., l \}.
\end{equation}


Let us take $\M_0$  --- the model with no constraints and $\M_{h_{ij}}$ --- the model under $h_{ij}$. For both models we are able to estimate the \emph{Maximum Likelihood Estimators} of group effects and denote them as~$\hat{\beta}_{\M_0}, \hat{\beta}_{\M_{h_{ij}}},$ respectively.

Then, the \textbf{Likelihood Ratio Test statistic} is

\begin{equation} \label{LRTstat}
LRT(\M_{h_{ij}}|\M_0) = 2 \cdot l (\hat{\beta}_{\M_0}|y) - 2 \cdot l (\hat{\beta}_{\M_{h_{ij}}}|y), 
\end{equation}

where $l(\cdot|y)$ is the log-likelihood function conditioned by observed $y$.

The higher the $LRT(\M_{h_{ij}}|\M_0)$, the more likely it is that constraint $h_{ij}$ is rejected. One may interpret the $LRT(\M_{h_{ij}}|\M_0)$ as a distance between group $i$ and $j$.

As $\M_{h_{ij}}$ for all $i,j$  ($i \neq j \;\; i,j \in \{1, 2, ..., l \}$) is nested in $\M_0$, the likelihood of $\M_{h_{ij}}$ is not greater than the $\M_0$'s likelihood. Therefore, if~$\mathcal{H}$~is a set of all considered constraint defined in~\eqref{constraint}, a constraint 

\begin{equation} \label{maxLikelihood}
\mathrm{argmin}_{h \in \mathcal{H}} LRT(\M_h|\M_0) = \mathrm{argmax}_{h \in \mathcal{H}} l(\M_h)
\end{equation}

will reduce the likelihood the least.

\paragraph{Asymptotic behaviour of the \emph{LRT statistic}}

An advantageous result by Samuel S. Wilks \citep{wilks1938large} shows that for a linear constraint $h$ the $LRT(\M_h|\M_0)$ tends asymptotically to chi-squared distribution with degrees of freedom equal to the difference in degrees of~freedom between $\M_0$ and $\M_h$ as number of observations approaches infinity. This convergence will be used to evaluate model's \emph{statistical correctness} in visualizations.



\subsection{Pairs considered}

In the Algorithm~\ref{outline} if we want to maximize the likelihood in the merging path we have to consider all feasible pairs while performing a single step. However, 
computing the objective function can be expensive and limiting the set of tested hypotheses is worth considering. Let us remark that it is more likely that a pair of levels $i$ and $j$ will be chosen to merge if corresponding estimators of effects ($\hat{\beta_i}, \hat{\beta_j}$) are close.

\todo{TODO: Czy to potrzebuje rozwinięcia?}

Therefore, in the package we implement two strategies of merging --- either comprehensive or limited. We call them:

\begin{itemize}
\item \emph{all-to-all} (with the argument \code{successive = FALSE}),
\item \emph{successive} (with the argument \code{successive = TRUE}).

\end{itemize}

The version \emph{all-to-all} considers all possible pairs of factor levels. In the \emph{successive} approach factor levels are preliminarily sorted and then only consecutive groups can be united. 

It is possible that the \emph{all-to-all} strategy will give better merging path, though, intuitively, the difference should not be significant.

\todo{TODO: Czy ja tak mogę pisać? Bez żadnego dowodu...}


\paragraph{The \emph{successive} merging}

In the \emph{successive} version of the algorithm at the very beginning levels of a categorical variable are sorted. The order depends on the model family chosen. In most cases it is associated with the beta coefficients estimators. The interpretation of beta coefficient is gathered in the table below.

\begin{table}[H]
\centering \begin{tabular}[t]{c|c}
\hline \textbf{model} & \textbf{metric} \\
\hline one-dimensional Gaussian & average \\
\hline multi-dimensional Gaussian & average of the isoMDS transformation \\
\hline binomial & proportion of successes \\
\hline survival & log hazard ratio \\
\hline 

\end{tabular}
\caption{\label{tab:}Factor ordering by model family}

\end{table}

For single dimensional Gaussian and binomial models groups are sorted by means and proportions of success, respectively. In the survival case we use log hazard ratios with one level set as the reference level. Multi dimensional Gaussian model needs additional preprocessing. First, group means are computed. Then they are projected into one dimension with the use of the Kruskal's non-metric multidimensional scaling. The \factorMerger uses \code{isoMDS} implementation from the package \textbf{MASS} \citep{MASS}. 

Having set the factor order, we may limit number of comparisons in each step in a way that a~particular level is tested only against its closest neighbours.



\subsection{Objective functions}

The \factorMerger package for each model family and merging strategy implements two types of a~single iteration of the algorithm. They use one of the following:

\begin{itemize}

\item \emph{Likelihood Ratio Test} (with the argument \code{method = "LRT"}),
\item \emph{agglomerative clustering with constant distance matrix} (based on the \emph{DMR4glm} algorithm, with the argument \code{method = "hclust"}). 

\end{itemize}




\subsection{The \emph{Likelihood Ratio Test}-based merging}

The \emph{Likelihood Ratio Test}-based approach minimizes likelihood reduction in the merging path. It may be summarized as follow.

\todo{TODO: Rozwinąć... (Analogia do LRT testów, ale można uprościć do samego loglik)}

\begin{algorithm}[H]
\caption{Merging with the $LRT$}
\begin{algorithmic}[2]

\Function{MergeFactors}{$response, factor, successive$}
\State{$pairsSet := generatePairs(response, factor, successive$)}
\State{$M_0:=$ full model}
\While{$levels(factor) > 1$}
    \State{$toBeMerged := \mathrm{argmax}_{pair \in pairsSet} l(updateModel(M_0, pair))$}
    
    \State{$M_0 := updateModel(M_0, toBeMerged)$}
    \State{$factor := mergeLevels(factor, pair)$}
    \State{$pairsSet := pairsSet \setminus pair $}
\EndWhile
    \EndFunction
\end{algorithmic}
\end{algorithm}

\subsection{The \emph{DMR4glm}-based merging}

\todo{TODO: Wstępny opis}

\begin{algorithm}[H]
\caption{Merging with agglomerative clustering}
\begin{algorithmic}[2]

\Function{MergeFactors}{$response, factor, successive$}
\State{$pairsSet := generatePairs(response, factor, successive$)}
\State{$dist :=$ set of distances }
\ForAll{$pair \in pairsSet$} 
    \State{$h := \{\mu_{pair_1} = \mu_{pair_2}\}$ }
    \Comment{hypothesis under which $pair$ is merged}
    \State{$dist$[$pair$] $= LRT(M_h|M_0)$}
\EndFor
        
\If{successive}
\State{$hClust$($dist$, method = "single")}
\Else 
\State{$hClust$($dist$, method = "complete")}
\EndIf
    \EndFunction
\end{algorithmic}
\end{algorithm}


\section{An \emph{R} package \factorMerger}

The \factorMerger package provides easy-to-use functions for factor merging and visualizing obtained results. 

TODO: Tutaj da się opis głównych funkcji - co robią i tabelki z opisem parametrów

\subsection{Merging and getting results}

\subsection{Visualizations}

TODO: Tutaj dojdzie jeszcze opis każdego z paneli osobno

\section{CASE STUDY: PISA2012}


\section{Summary}

Some summary. 
\section{Acknowledgements}

The authors thank to ... .

Work on this package was financially supported by the \emph{NCN Opus grant 2016/21/B/ST6/02176}.


\bibliography{sitko_biecek}

\address{Agnieszka Sitko \\
  University of Warsaw\\
  Faculty of Mathematics, 
  Informatics and Mechanics\\
  Poland \\
  \email{ag.agnieszka.sitko@gmail.com}}

\address{Przemysław Biecek \\
  University of Warsaw \\
  Institute of Applied Mathematics and Mechanics\\
  Poland\\
  \email{P.biecek@mimuw.edu.pl}}

\documentclass[11pt,]{article}
\usepackage[left=1in,top=1in,right=1in,bottom=1in]{geometry}
\newcommand*{\authorfont}{\fontfamily{phv}\selectfont}
\usepackage[]{mathpazo}


  \usepackage[T1]{fontenc}
  \usepackage[utf8]{inputenc}



\usepackage{abstract}
\renewcommand{\abstractname}{}    % clear the title
\renewcommand{\absnamepos}{empty} % originally center

\renewenvironment{abstract}
 {{%
    \setlength{\leftmargin}{0mm}
    \setlength{\rightmargin}{\leftmargin}%
  }%
  \relax}
 {\endlist}

\makeatletter
\def\@maketitle{%
  \newpage
%  \null
%  \vskip 2em%
%  \begin{center}%
  \let \footnote \thanks
    {\fontsize{18}{20}\selectfont\raggedright  \setlength{\parindent}{0pt} \@title \par}%
}
%\fi
\makeatother




\setcounter{secnumdepth}{0}

\usepackage{longtable,booktabs}


\title{factorMerger: a set of tools to support results from post hoc testing  }



\author{\Large Agnieszka Sitko\vspace{0.05in} \newline\normalsize\emph{University of Warsaw}  }


\date{}

\usepackage{titlesec}

\titleformat*{\section}{\normalsize\bfseries}
\titleformat*{\subsection}{\normalsize\itshape}
\titleformat*{\subsubsection}{\normalsize\itshape}
\titleformat*{\paragraph}{\normalsize\itshape}
\titleformat*{\subparagraph}{\normalsize\itshape}


\usepackage{natbib}
\bibliographystyle{apsr}



\newtheorem{hypothesis}{Hypothesis}
\usepackage{setspace}

\makeatletter
\@ifpackageloaded{hyperref}{}{%
\ifxetex
  \usepackage[setpagesize=false, % page size defined by xetex
              unicode=false, % unicode breaks when used with xetex
              xetex]{hyperref}
\else
  \usepackage[unicode=true]{hyperref}
\fi
}
\@ifpackageloaded{color}{
    \PassOptionsToPackage{usenames,dvipsnames}{color}
}{%
    \usepackage[usenames,dvipsnames]{color}
}
\makeatother
\hypersetup{breaklinks=true,
            bookmarks=true,
            pdfauthor={Agnieszka Sitko (University of Warsaw)},
             pdfkeywords = {analysis of variance (ANOVA), hierarchical clustering, likelihood ratio
test (LRT), post hoc testing},  
            pdftitle={factorMerger: a set of tools to support results from post hoc testing},
            colorlinks=true,
            citecolor=blue,
            urlcolor=blue,
            linkcolor=magenta,
            pdfborder={0 0 0}}
\urlstyle{same}  % don't use monospace font for urls



\begin{document}
	
% \pagenumbering{arabic}% resets `page` counter to 1 
%
% \maketitle

{% \usefont{T1}{pnc}{m}{n}
\setlength{\parindent}{0pt}
\thispagestyle{plain}
{\fontsize{18}{20}\selectfont\raggedright 
\maketitle  % title \par  

}

{
   \vskip 13.5pt\relax \normalsize\fontsize{11}{12} 
\textbf{\authorfont Agnieszka Sitko} \hskip 15pt \emph{\small University of Warsaw}   

}

}







\begin{abstract}

    \hbox{\vrule height .2pt width 39.14pc}

    \vskip 8.5pt % \small 

\noindent \textbf{factorMerger} is an \emph{R} package whose purpose is to extend
methods of analysing dependencies between groups of a categorical
variable after carrying out an analysis of variance (ANOVA). The idea of
the package arose from the need to create an algorithm which outputs in
a hierachical interpretation of relations between levels of a
categorical variable. Thereby, for a given significance level groups may
be devided into nonoverlapping clusters. \textbf{factorMerger}
implements iterative version of post hoc testing based on likelihood
ratio test for parametric models: gaussian, binomial and survival. It
also provides custom visualizations for each model built on
\textbf{ggplot2} package. ~\\
\emph{Package webpage}:
\url{https://github.com/geneticsMiNIng/FactorMerger}


\vskip 8.5pt \noindent \emph{Keywords}: analysis of variance (ANOVA), hierarchical clustering, likelihood ratio
test (LRT), post hoc testing \par

    \hbox{\vrule height .2pt width 39.14pc}



\end{abstract}


\vskip 6.5pt

\noindent  \section{Introduction}\label{introduction}

If data is analysed using ANOVA a more detailed analysis of differences
among categorical variable levels might be needed. The traditional
approach will be to perform \emph{pairwise post hocs} - multiple
comparisons after ANOVA. For each pair of groups we run specific
statistical test which outputs with an information whether response
averages in those groups are significantly different. However, if we
look from the distant perspective, for a certain significance level,
these results may not be consistent. One may consider the case that mean
in group A does not differ significantly from the one in group B,
similarly with groups B and C. In the same time difference between group
A and C is detected. Then data partition is unequivocal and, as a
consequence, impossible to put through.

The problem of clustering categorical variable into non-overlapping
groups has already been present in statistics. First, J. Tukey proposed
an iterative procedure of merging factor levels based on studentized
range distribution \citep{Tukey}. However, statistical test used in this
approach made it limited to gaussian models. Collapse And Shrinkage in
ANOVA (CAS-ANOVA, \citet{Casanova}) is an algorithm that extends
categorical variable partitioning for generalized linear models. It is
based on the Tibshirani's Fused LASSO \citep{Tib} with the constraint
taken on the pairwise differences within a factor, which yields to their
smoothing.

Delete or Merge Regressors algorithm \citep{Proch} is also adjusted to
generalized linear models. It directly uses hierarchical clustering to
gain hierarchical structure of a factor. At the beginning,
\emph{DMR4glm} calculates likelihood ratio test statistics for models
arising from pairewise merging of factor levels against the initial
model (the one with all groups included). Then performs agglomerative
clustering taking LRT statistic as a distance --- each step of
clustering is associated with model with different factor structure.
Experimental studies \citep{ProchEx} showed that Delete or Merge
Regressors's performance is better than CAS-ANOVA's when it comes to
model accuracy.

\textbf{factorMerger} package gives an approximate implementation of
\emph{DMR4glm}. In addition to the base algorithm, it also provides its
sequential version, which merges only those levels which are relatively
close (levels distance is dependent on the model chosen). While the
basic approach (all vs.~all comparisons) may sometimes result in a
slightly better partition from the statistical point of view, proposed
extention (all vs.~subsequent comparisons) seems to be more graceful
when it comes to the interpretation. Moreover, the former is more
computationally expensive.

Furthermore, \textbf{factorMerger} offers yet another algorithm of
hierarchical clustering. This is also an iterative procedure, but in
each step it chooses model with the highest likelihood. While this
algorithm is more complex than \emph{DMR4glm} it is easily expandable
for non-parametric models (using permutation tests instead of LRTs). The
greedy algorithm is also available in two versions - comprehensive and
sequential.

More detailed description of all algorithms implemented in
\textbf{factorMerger} is given in the section \emph{Algorithms
overview}.

\section{Algorithms overview}\label{algorithms-overview}

\subsection{Sequential version}\label{sequential-version}

In the sequential version of the algorithm at the begining categorical
variable is releveled. Depending on the model family chosen, we specify
different statistics to set levels order.

\begin{longtable}[]{@{}cc@{}}
\toprule
model & metric\tabularnewline
\midrule
\endhead
single dimensional gaussian & mean\tabularnewline
multi dimensional gaussian & mean of isoMDS fit\tabularnewline
binomial & success proportion\tabularnewline
survival & relative survival coefficient\tabularnewline
\bottomrule
\end{longtable}

For single dimensional gaussian and binomial models groups are sorted by
means and proportions of success, respectively. In survival case we
estimate survival model which takes all factor levels separately. Then
beta coefficient approximations specify levels order (base level gets
coefficient equal to zero). Multi dimensional gaussian model needs
additional preprocessing. We propose to order levels by means of isoMDS
projection (into one dimension, currently isoMDS from package
\textbf{MASS} is used). However, the projection is used only in this
preliminary stage. In the merging phase of the algorithm all test
statistics are calculated for multi dimensional gaussian model.
Calculating isoMDS projection is an expensive procedure --- it usually
takes more time than the merging phase.

Having set the factor order, we may limit number of comparisons in each
step.

\subsection{DMR4glm}\label{dmr4glm}

\subsection{Greedy algorithm}\label{greedy-algorithm}

\subsection{Cost comparisons}\label{cost-comparisons}

\section{\texorpdfstring{The \emph{R} package
\textbf{factorMerger}}{The R package factorMerger}}\label{the-r-package-factormerger}

\subsection{Setting up merging
options}\label{setting-up-merging-options}

\subsection{Visualizations}\label{visualizations}

\subsection{Sample results}\label{sample-results}

\section{Possible extensions}\label{possible-extensions}

\newpage
\singlespacing 
\renewcommand\refname{Bibliography}
\bibliography{factorMerger.bib}

\end{document}
